\section{High luminosity LHC}
\label{sec:hllhc}

Over the next twenty years, the LHC plans to steadily increase the rate at which proton-proton collisions are delivered to the ATLAS and CMS experiments. The final design of the machine, called the High Luminosity LHC, could potentially deliver a peak instantaneously luminosity of $\mathcal{L} = 7.5\times10^{34}\text{ Hz/cm}^2$. This is almost ten times larger than the original design luminosity of the LHC of $\mathcal{L} = 1\times10^{34}\text{ Hz/cm}^2$~\cite{lhc}. It is clear that the predicted instantaneously luminosity is an estimate, and experiments would be prudent to consider the effect of even higher luminosity.

To predict the hit rate in the NSW at the HL-LHC, the hit rate in the current Small Wheel is measured and extrapolated to higher luminosity~\cite{tuna,phase2}. The measured hit rate increases linearly with the luminosity, as expected, giving confidence to the extrapolation. The current SW is comprised of MDT and CSC detectors, and detector effects are factored out to give a detector-independent prediction for the NSW.

The prediction is shown in Fig.~\ref{fig:rate_vs_r} as a function of $R$, the radial distance from the beamline. On the left is the predicted rate per unit area, and on the right is the predicted rate per Micromegas strip. For simplicity, some results in this note are presented with an uncorrelated background rate of 40 kHz per strip, which corresponds to the rate prediction near the beamline. For prudence, other results are presented as a function of the hit rate.
\begin{figure}[!htpb]
  \begin{center}
    \includegraphics[width=0.48\textwidth]{figures/ilia.pdf}
    \includegraphics[width=0.48\textwidth]{figures/rate_per_strip.pdf}
  \end{center}
  \vspace{-10pt}
  \caption{Predicted rate vs. distance from the beamline at $\mathcal{L} = 7.5\times10^{34}\text{ Hz/cm}^2$, per unit area (left) and per strip (right). The rate per area is culled from the Phase 2 Muon TDR~\cite{phase2}. The rate per strip is derived from the rate per area and the expected area of the Micromegas chambers.}
  \label{fig:rate_vs_r}
\end{figure}

